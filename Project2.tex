\documentclass[12pt]{article}
\usepackage{geometry}                % See geometry.pdf to learn the layout options. There are lots.
\geometry{letterpaper}                   % ... or a4paper or a5paper or ... 
%\geometry{landscape}                % Activate for for rotated page geometry
%\usepackage[parfill]{parskip}    % Activate to begin paragraphs with an empty line rather than an indent
\usepackage{graphicx}
\usepackage{amssymb}

\usepackage{fontspec,xltxtra,xunicode}
\defaultfontfeatures{Mapping=tex-text}

\usepackage{xeCJK}
\setCJKmainfont[BoldFont=STSong, ItalicFont=STKaiti]{STSong}
\setCJKsansfont[BoldFont=STHeiti]{STXihei}
\setCJKmonofont{STFangsong}
\title{个性化推荐}
\author{张岚}
%\date{}                                           % Activate to display a given date or no date

\usepackage{algorithm}
\usepackage{algorithmicx}
\usepackage{algpseudocode}
\usepackage{amsmath}

\floatname{algorithm}{算法}
\renewcommand{\algorithmicrequire}{\textbf{输入:}}
\renewcommand{\algorithmicensure}{\textbf{输出:}}

\begin{document}
\maketitle

\begin{enumerate}
\item[1.] 使用全量数据,取index为用户id,column为电影id,从训练集中取对应的用户id对电影id的评分,建立一个新的矩阵,对于未知的项全定为0,见算法1。
 \begin{algorithm}
        \caption{数据预处理}
        \begin{algorithmic}[1] %每行显示行号
            \Require $data$数据
            \Function {procData}{$data$}
                \State $uids \gets data.uid.unique$
                \State $fids \gets data.fid.unique$
                \State $df \gets zeros(uids,fids)$
                \For{$line \in data$}
                	    \State $x \gets line[1]$
	    	    \State $y \gets line[2]$
                     \State $df[x,y]  \gets line[3]$
                \EndFor
                \State \Return{$df$}
            \EndFunction    
        \end{algorithmic}
\end{algorithm}

\item[2.] 使用训练集的数据计算用户的相似度,预测测试集中用户对电影的打分,最后评估准确率,RMSE值为0.772586199418。本文使用矩阵形式运算程序运行CPU耗时32.288,使用矩阵形式和协同过滤方法见算法2。
 \begin{algorithm}
        \caption{协同过滤}
        \begin{algorithmic}[1] %每行显示行号
            \Require $X\_train$训练数据,$X\_test$测试数据
            \Function {Task1}{$X\_train,X\_test$}
                \State $sim \gets $ \Call{$cosine\_similarity$}{$X\_train$}
                \State $pred \gets similarity.dot(X\_train) / np.array([np.abs(similarity).sum(axis=1)]).T$
                \State $print  \gets RMSE((X\_test,pred))$
            \EndFunction    
        \end{algorithmic}
\end{algorithm}

\item[3.] 矩阵分解算法如下。
 \begin{algorithm}
        \caption{矩阵分解}
        \begin{algorithmic}[1] %每行显示行号
            \Require $X_train$训练数据,$X_test$测试数据
            \Function {Task1}{$X_train,X_test$}
                \State $sim \gets cosine_similarity(X_train)$
                \State $pred \gets similarity.dot(X_train) / np.array([np.abs(similarity).sum(axis=1)]).T$
                \State $print  \gets RMSE((X_test,pred))$
            \EndFunction    
        \end{algorithmic}
\end{algorithm}

\begin{enumerate}
\item[a.] 给定 $k=50$, $\lambda = 0.01$,不同$\alpha$得到的目标函数值和测试集上 RMSE变化见下图。
\item[b.] 不同k值对RMSE的影响,不同$\lambda$对RMSE的影响,程序最终选择。


\item[4.] 计算时间上,准确率上
\end{enumerate}
\end{enumerate}
\end{document}  